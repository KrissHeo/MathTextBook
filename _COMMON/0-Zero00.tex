\mychapter{교재 구성 및 활용법}{}

\section{교재 소개 : 단원을 넘나들며 함수를 중심으로 학습하는 전범위 통합개념서}
이 책의 제목은 \cnm{인투더 수학 I \& II}이지만, 사실 더 적합한 제목은 \cnm{인투더 함수}라 할 수 있을 정도로, 모든 내용은 함수를 중심으로 편성되어 있습니다. 고등학교 1학년 \cnm{수학}은 모두 함수를 다루기 위한 재료들을 준비하는 단계이고, \cnm{수학 I}은 새로운 함수들을 배우는 과목이며, \cnm{수학 II}는 익숙한 함수였던 다항함수를 미적분이라는 도구를 이용하여 분석하는 과목이기 때문입니다. 즉 \cnm{수학 I}과 \cnm{수학 II}의 전범위 개념을 완전히 장악하기 위해서는 그 공통분모인 함수에 대한 개념을 정확히 확립해야 한다고 볼 수 있습니다.

대부분의 다른 수학 개념 교재들은 물론이고, 인투더 시리즈의 다른 책들인 \cnm{인투더 중학도형}, \cnm{인투더 확률과 통계}, \cnm{인투더 기하}, \cnm{인투더 미적분}도 대체로 단원순서대로 공부하는 구성을 따르고 있습니다. 이에 반해 \cnm{인투더 수학 I \& II}는 \cnm{수학}, \cnm{수학 I}, \cnm{수학 II}를 넘나들며 함께 다룰 수 있는 주제들을 묶어가며 가르칩니다. 통합수능 체제에서 \cnm{수학 I}과 \cnm{수학 II}가 통합된 문제를 출제하며 수험생들에게 한 단원에 갇힌 좁은 시야에서 벗어나기를 요구한다는 점에서, 이러한 구성의 교재를 학습하는 것은 여러분께 넓게 바라보는 안목을 갖출 수 있도록 도울 것입니다.
\section{교재 구성}
이 책은 다음과 같이 구성되어 있습니다. 
\begin{itemize}
    \item Zero : \cnm{수학}, \cnm{수학 I}, \cnm{수학 II}의 기본 용어
    \item Graph : 함수의 그래프와 관련된 개념의 시각화
    \item Basic : Zero와 Graph의 내용을 수식으로 엄밀화
    \item Algebra : 2015 개정 교육과정 수학 I (2022 개정 교육과정 대수-Algebra)
    \item Calculus : 2015 개정 교육과정 수학 II (2022 개정 교육과정 미적분 I-Calculus)
\end{itemize}
뒷 단원은 앞 단원의 내용을 숙지했다는 전제 하에 내용이 구성되어 있습니다. 따라서 뒷 단원을 먼저 읽는 것은 권장하지 않습니다. 


%용어 파트와 원리 파트를 공부함으로써, 쎈수학 A단계와 B단계 난이도(하), 교과서 본문에 제시되는 예제와 유제 수준, 수능에서는 모든 2점 문항과 쉬운 3점 문항을 스스로 풀이할 기초를 닦을 수 있습니다.

%이해 파트를 공부함으로써, 쎈수학 B단계 난이도(중), 교과서 단원 후반에 제시되는 문항 수준, 수능에서는 일반적인 3점 문항과 쉬운 4점 문항을 풀이하는 데 필요한 소양을 갖출 수 있습니다.

%응용파트를 공부함으로써, 쎈수학 B단계 난이도(중, 상)과 C단계, 교과서에 실린 고난도 문항이나 수능 4점 문항에 대응할 준비를 마칠 수 있습니다.

\section{교재 활용법 : 다회독을 통한 개념 구조의 확립}
뒤이어 설명할 단원별 공부법을 참고하여 1회독을 마친 뒤, 2회독부터는 1회독에서 이미 알고 있던 내용이나 쉽다고 느꼈던 부분을 건너뛰며 학습합니다. 익숙하지 않았거나 완전히 이해되지 않은 부분만 집중적으로 학습하며 약점을 메웁니다. 회독을 거듭할 수록, 교육과정 전반에서 다루는 함수의 기초부터 핵심까지 뚜렷한 이미지를 확립할 수 있을 것입니다.

\section{단원별 공부법}

\subsection{Zero : 가능한 빠르게 읽고, 빈 부분이 있다면 체크하자}
Zero는 본 교재를 공부하기 위해 꼭 알고 있어야 하는 교과서 기본 개념을 담고 있습니다. 단순히 교육과정 순서대로 나열한 것이 아니라, 기본개념을 한 번 이상 공부한 학생이 12년치의 교과 개념을 가장 빠르고 효율적으로 재확인할 수 있도록 최적화하였습니다.

Zero를 읽어나가면서 의구심이 들거나 알 듯 말 듯 애매한 개념들이 있는지를 확인해야 합니다. 그런 게 없다면 개념이 탄탄함을 확인해서 좋은 것이고, 있다면 이를 보완할 수 있게 되어 다행이라고 생각합시다. 단, Zero에서 저자의 판단에 따라 상세한 설명이나 증명을 제시하는게 필요한 경우에는 제시되어 있지만, 대부분의 경우는 증명이나 상세한 설명이 없습니다. 그러한 경우에는 교과서나 참고서를 활용하여 해당 내용을 확인하여 개념을 보완하시기 바랍니다.

학생에 따라서는 어차피 쉬운 개념이고 단순한 나열인데 Zero 공부 안하고 뛰어넘으면 안되냐고 생각할 수 있겠습니다. 그러나 교과서에서는 본 적이 없는 서술, 표현, 용어 등이 약간씩 들어있으므로, 원활한 학습을 위해서는 Zero를 꼭 공부해야 합니다.  또한 이 책의 본문은 Zero를 다 읽었다는 전제 하에 서술되므로, 이 책을 효과적으로 공부하기 위해서는 Zero를 반드시 학습하기를 권합니다.

\subsection{Graph : 개념의 정의를 암기하고, 정의에서 파생된 성질을 시각화한 내용에 익숙해지자}
함수는 수식으로 다루는 것도 중요하지만, 그래프를 통해 시각화된 내용에 친숙해야 합니다. 이는 수능이 엄밀성을 요구하는 서술형 시험이 아닌, 직관과 비약을 어느 정도 허용하는 단답형 시험이기 때문에 더욱 그렇습니다. 이를 위해 Graph에서는 함수와 관련된 여러 가지 정의와 성질들을 알려주고, 정의와 성질로부터 끌어낼 수 있는 여러 상황들을 증명 없이 제시합니다. 따라서 각 정의를 확실히 암기하고, 성질을 이해하고, 시각화된 결과물들에 친숙해지는 것이 필요합니다. \clearpage

\subsection{Basic : 함수의 엄밀화}
Basic은 Zero와 Graph까지만 알고 있는 상황에서 본격적인 함수 분석을 다루기 전 이야기할 수 있는 함수 관련 내용을 주제별로 엮어 배웁니다. Basic을 배울 때에는 앞서 배운 내용들이 계속 복선처럼 활용되므로, 반복 학습을 통해 익숙해질 수 있도록 도와줍니다.

\subsubsection{Basic 1) 수식으로 다루는 함수의 성질}
지금까지 배운 내용을 수식으로 검증하는 단원입니다. 아무리 수능이 직관적이거나 감각적인 영역이 강조된다고 하더라도, 수식을 아예 배제할 수는 없습니다. 그래서 Graph로 여러 정의와 성질에 익숙해졌다면, 이제 수식과 논리로 바탕을 든든히 해야 할 것입니다. Basic 1)을 통해 Graph에서 배운 내용을 더 깊이 이해할 수 있고, 확실하게 익숙해질 수 있을 것입니다. 수식 활용 능력과 논리를 익히는 건 보너스입니다.


\subsubsection{Basic 2) 점근선과 극한의 논리 완성}
어물쩡 대충 계산하고 넘어가는 극한이 아닌, 논리에 근거해 깔끔하게 풀이하는 극한을 배웁니다. `대충 뭐가 어디로 가고 어디로 가니까 뭐가 어떻고$\cdots$'와 같이 확실하지 않은 애매한 풀이를 해왔다면, 이 중단원을 통해 논리적으로 극한을 풀이할 수 있을 것입니다.

\subsubsection{Basic 3) 여러 가지 함수의 분석}
교육과정 순서대로 각 함수의 특징을 되돌아봅니다. 또한 지금까지 배운 내용을 바탕으로 다항함수를 두 소단원에 걸쳐 분석하면서 다항함수에 대해 더 깊게 이해할 수 있습니다.

\subsection{Algebra : 더 많은 함수들과 함수로서의 수열(수학 I)}
Algebra에서는 지금까지 다룬 내용들을 바탕으로 \cnm{수학 I}을 다시 살펴봅니다. 많은 내용들을 알게 모르게 앞에서 다루어왔으므로, Algebra 단원에서는 쉬어간다는 느낌으로 앞에서 배운 내용들을 언급하며 가볍게 개념들을 리마인드하고, 각 개념에 대해 새롭게 해석하는 관점들을 제공합니다.


\subsection{Calculus : 미적분을 통한 다항함수의 분석과 합성함수 특강(수학 II)}
지금까지 배운 내용을 모두 종합하여 다항함수의 미적분을 완성하고, 미처 다루지 못한 미분계수, 적분의 의미, 수직선에서의 물리학을 다룹니다. 마지막으로 미적분 미선택자의 불리함을 완화하는 특강으로 마무리짓습니다.


\begin{comment}
\section{1권 : 문제는 거들 뿐, 이론 확립을 위해 개념 위주로 학습합니다.}
이 책은 수능에 필요한 대부분의 미적분 이론을 한 권에 꾹꾹 눌러담는 것을 목표로 했기 때문에, 1권 본문에는 일부 예외적인 경우를 제외하고는 수록된 문제가 없습니다. 이 책의 예상 독자가 `미적분 전범위를 어느 정도 공부한 경험이 있는 상태에서 개념을 체계적으로 정립하려는 학생'이므로, 수록된 문제가 많지 않다는 것만 미리 알고 있으면 학습에 큰 지장이 없을 것입니다.

\section{2권 : 2005\anfruf2011 \cnm{미적분} 기출문제로 문제풀이에 적용하세요.}
1권에는 개념만 있고 문제풀이가 없는데 1권만 공부하면 개념이 탄탄해졌으니 수능 점수가 잘 나오느냐? 그것은 아닙니다. 1권에서 배운 이론을 기출문제에 실제로 사용해보고 정말 내가 이론을 잘 알고 있는지, 문제풀이에는 잘 적용할 수 있는지 확인하고 연습해야겠죠. 

이는 2권에서 다룹니다. 1권의 내용을 바탕으로, 2005학년도\textasciitilde{}2011학년도의 평가원/수능 \cnm{미적분} 주요 기출문제들을 연도순서대로 분석합니다. 수능이론의 내용이 기출문제에 어떻게 접목되는지를 설명함과 동시에, 일관성을 유지하면서도 꾸준히 변형/발전하는 기출문제의 출제 경향을 생생하게 배울 수 있을 것입니다.

2005\anfruf{}2011 \cnm{미적분} 기출문제 중 빠진 문항들과 2012\anfruf{}2021학년도의 \cnm{미적분} 평가원/수능 기출문제들은 12월에 출간될 \cnm{트리플기출}에서 다룹니다. 특수주제 문항(극한-도형 응용, 변화율 등)을 제외하고는 2권과 마찬가지로 연도순서대로 분석하므로 일관적인 기출 학습이 가능할 것입니다.


\mychapter{단원별 공부법}{}
\section{단원별 명명법}
단원의 이름을 부르는 법부터 약속합시다. 단원은 대중소로 나뉘는데, 대단원은 Zero, Integral, Function, Limit, Special의 5개로 나뉩니다. Limit 1)과 Function 3)은 중단원을 부르는 이름입니다. Zero 2.1)과 Special 3.2)는 소단원을 부르는 이름입니다.

\section{Zero : 가능한 빠르게 읽고, 빈 부분이 있다면 체크하자}
Zero는 본 교재를 공부하기 위해 꼭 알고 있어야 하는 \cnm{미적분}의 교과서 기본 개념을 담고 있습니다. Zero를 읽어나가면서 의구심이 들거나 알 듯 말 듯 애매한 개념들이 있는지를 확인해야 합니다. 그런 게 없다면 개념이 탄탄함을 확인해서 좋은 것이고, 있다면 이를 보완할 수 있게 되어 다행이라고 생각합시다. 단, Zero에서 저자의 판단에 따라 상세한 설명이나 증명을 제시하는게 필요한 경우에는 제시되어 있지만, 대부분의 경우는 증명이나 상세한 설명이 없습니다. 그러한 경우에는 교과서나 참고서를 활용하여 해당 내용을 확인하여 개념을 보완하시기 바랍니다.

학생에 따라서는 어차피 쉬운 개념이고 단순한 나열인데 Zero 공부 안하고 뛰어넘으면 안되냐고 생각할 수 있겠습니다. 그러나 교과서에서는 본 적이 없는 서술, 표현, 용어 등이 들어있으므로, 원활한 학습을 위해서는 Zero를 꼭 공부해야 합니다.  또한 이 책의 본문은 Zero를 다 읽었다는 전제 하에 서술되므로, 이 책을 효과적으로 공부하기 위해서는 Zero를 반드시 학습하기를 권합니다.

\section{Integral : 적분 테크닉의 연마}
\cnm{수학 II}의 적분과 \cnm{미적분}의 적분의 가장 두드러진 차이점은 함수를 적분하는 데 여러 테크닉이 필요하다는 점입니다. 본질을 잊지 않은 채 각 테크닉을 배워나갈 수 있도록 합니다.

\subsection{Integral 1) 기본적인 적분 테크닉}
단순히 Case by case식의 접근이 아닌, 각 테크닉의 본질적 요소를 탐구하여 적분 테크닉의 기본기를 탄탄히 다집니다. 그냥적분, 치환적분, 부분적분을 배우고, 본질에 입각하여 다시 생각하면 테크닉들은 그저 수단일 뿐이라는 점을 다시 한번 상기시켜 줍니다.

\subsection{Integral 2) 고급진 적분 테크닉}
기본기를 다진 후 구사할 수 있는 고급진 적분 스킬을 배우며 원시함수를 구하는 다양한 적분 테크닉을 연마합니다.
\clearpage

\section{Function : 함수와 관련된 종합선물세트}
학생들이 가장 다루기 힘들어하는 합성함수와 역함수를 정복합니다. 그 이후 미분법을 통해 그래프를 그리는 연습을 하며 기본기를 다집니다. 

\subsection{Funciton 1) $f\left( g\left( x \right)  \right) $와 $f\left( f\left( x \right)  \right) $ 그리기}
합성함수 $f\left( g\left( x \right)  \right) $와 $f\left( f\left( x \right)  \right) $를 논리적으로 그리는 방법을 배움으로써 합성함수에 대한 두려움을 없애줍니다.

\subsection{Function 2) 역함수에 대한 탐구와 유사 역함수 조건}
교과서의 실책으로 인한 역함수에 대한 혼동을 없애 역함수에 대한 개념을 확실히 잡습니다. 그 후 역함수와 역함수 미분법, 유사 역함수 조건에 대한 탐구를 통해 역함수 관련 내용을 탄탄히 다집니다.

\subsection{Function 3) 미분을 통해 그래프 그리기}
미분법을 통해 그래프를 그리는 기본기를 다지고 더하기함수, 빼기함수, 곱하기함수, 나누기함수를 그리는 방법을 배워봅니다.

\subsection{Function 4) 음함수와 매개변수 길들이기}
음함수와 매개변수에 대해 무엇이 출제 가능하고 무엇이 불가능한지를 철저하게 따지고, 음함수와 매개변수, 특히 매개변수에 대해 탐구합니다. 

\subsection{Function 5) 물리학, 그리고 벡터}
위치, 속도, 가속도, 거리를 미적분으로 해석합니다. 5.1)에서는 교과 내의 설명으로 1페이지만에 깔끔하게 정리합니다. 5.1)만으로도 수능을 준비하는데는 아무런 문제가 없으므로 5.2)와 5.3)은 공부하지 않아도 무방합니다. 5.2)에서는 교과서의 허술한 서술을 보강하기 위해 벡터를 배우고, 5.3)에서는 벡터로 물리학을 해석함으로써 물리학에 대한 수학적 표현이 왜 그렇게 나타나는지를 명쾌하게 설명합니다.
\clearpage
\section{Limit}
\subsection{초월함수의 극한}
초월함수의 극한을 다시 살펴봅니다.
\subsection{극한과 도형이 만났을 때}
도형에 극한을 응용하는 문제들의 해법을 알아봅니다. 
\subsection{구분구적법과 정적분의 진실}
구분구적법을 배워보고 정적분의 진실을 배워봅시다.

\section{Special : 굳이 원한다면 주제별 특강을 통해 궁금증을 해결하자}
Special에서는 세 가지 특별한 주제를 다룹니다. 여기서 배우는 내용들은 많은 학생들이 공통적으로 궁금해하는 주제이다보니 매년 반복적으로 화제가 되는 이슈들입니다.

\subsection{Special 1) 역함수와 합성함수 확장팩}
Function의 심화학습을 위한 단원입니다. Special 1.1)에서는 역함수를 해석하기 위한 새로운 용어를 배우고, 이를 바탕으로 역함수 존재 조건과 유사 역함수 조건을 해석해봅니다. Special 1.2)에서는 매나함을 통해 합성함수를 해석하고, 벡터를 이용하여 합성함수의 그래프를 그리는 방법을 배웁니다. 

\subsection{Special 2) 볼록성과 접할선}
볼록성이 할선과 접선의 위치관계에 미치는 영향과 접선의 개수에 대한 내용을 단계적으로 배워봅니다.

\subsection{Special 3) 판도라의 상자}
도함수의 극한과 미분계수의 정의가 미묘하게 다르다는 점 때문에 빚어지는 오개념이 있습니다. 이 오개념을 모르고 편하고 행복한 삶을 살 것인지, 호기심을 이기지 못하고 진실이 궁금해 판도라의 상자를 열어젖힐 지는 학생 여러분의 선택입니다.
\end{comment}

\begin{comment}

\mychapter{교재 활용 시 유의사항}{}
\section{공부하기 전 반드시 알아두어야 할 사항}
\subsection{교재 정오표 확인, 교재 내용 질문을 위해선 꿀탐(ggultam.com)에 방문해주세요!} 
오류가 없는 교재를 만들기 위해 최선을 다했지만 미처 잡지 못한 정오사항이 있을 수 있습니다. 꼭 공부하기 전 정오사항 여부를 확인하고 있다면 반영한 후 공부를 시작하세요. 또한 이 책으로 공부하다가 내용, 문제, 해설에 궁금한 점이 있다면 주저하지 말고 꿀탐에 방문하세요. 저자가 가급적 빠른 시일 내에 답변을 달아드립니다. 본 교재의 내용을 다른 문제집에 어떻게 적용해야 하는지 질문하셔도 좋습니다. 저희가 여러분과 함께 고민해드리겠습니다. 

\subsection{헷갈리는 용어는 색인을 통해 용어 설명 페이지를 찾을 수 있습니다}
    책의 맨 뒤에 색인이 있습니다. 헷갈리는 용어나 처음 보는 용어가 있다면 색인을 이용하여 모르는 용어를 설명하고 있는 페이지를 찾아가세요. 교육과정 내의 용어는 굵은 글씨에 검정색으로 표시되어 있고, 교육과정에 없거나 이 책에서 새로이 정의한 용어는 굵은 글씨에 분홍색으로 표시되어 있습니다.

\section{인투더에서는 문제풀이는 그저 거들 뿐이며, 이론 확립을 위해 개념 위주로 학습합니다.}
인투더는 수능에 필요한 대부분의 수학 I \&{} 수학 II 이론을 한 권에 꾹꾹 눌러담는 것을 목표로 했습니다. 따라서 본문에는 일부 예외적인 경우를 제외하고는 수록된 문제가 없습니다. 이 책의 예상 독자가 `{\cnm{수학 I}}과 \cnm{수학 II} 전범위를 어느 정도 공부한 경험이 있는 상태에서 개념을 체계적으로 정립하려는 학생'이므로, 수록된 문제가 많지 않다는 것만 미리 알고 있으면 학습에 큰 지장이 없을 것입니다. 

\section{트윈기출로 2005\anfruf{}2021 평가원 기출문제를 풀며 문제풀이에 적용하세요.}
인투더만 열심히 공부하면 개념이 탄탄해졌으니 수능 점수가 잘 나오느냐? 그것은 아닙니다. 인투더에서 배운 이론을 기출문제에 실제로 사용해보고 정말 내가 이론을 잘 알고 있는지, 문제풀이에는 잘 적용할 수 있는지 확인하고 연습해야겠죠. 

이는 트윈기출에서 다룹니다. 인투더의 내용을 바탕으로, 2005\anfruf{}2021 \cnm{수학 I} 기출문제들은 단원별로, 2005\anfruf{}2021학년도의 \cnm{수학 II} 기출문제들은 연도별로 다룹니다. 인투더의 내용이 기출문제에 어떻게 접목되는지를 설명함과 동시에, 일관성을 유지하면서도 꾸준히 변형/발전하는 기출문제의 출제 경향을 생생하게 배울 수 있을 것입니다.





\mychapter{단원별 공부법}{}
\section{단원별 명명법}
단원의 이름을 부르는 법부터 약속합시다. 단원은 대중소로 나뉘는데, 대단원은 Zero, Graph, Basic, Math I, Calc의 5개로 나뉩니다. Graph 1)과 Math I 3)은 중단원을 부르는 이름입니다. Calc 1.3)과 Basic 3.2)는 소단원을 부르는 이름입니다.


\section{미적분 내용은 따로 표시되어 있습니다}
이 책은 \cnm{미적분}을 선택하지 않은 수험생을 대상으로 제작되어 있으나, \cnm{미적분}을 선택한 학생들의 학습 편의를 위해 소소하게 배려된 부분이 있습니다. 중단원이 \cnm{미적분} 내용인 경우는 회색 바탕으로 시작하므로 적절히 건너뛰면 됩니다. 소단원 서술 중간에 \cnm{미적분} 선택자들만 봐야 하는 내용은 주석이나 (미적분 선택자 전용) 표시를 통해 구별되어 있습니다. \cnm{미적분} 선택자들은 처음부터 끝까지 순서대로 쭉 학습하면 됩니다.


\section{Function : 합성함수와 역함수를 확실하게 이해하자}
많은 학생들이 괴로워하는 합성함수, 역함수를 `미적분을 안다는 전제 하에' 상세하고 명쾌하게 설명하였습니다. 이 이상으로 합성함수와 역함수에 대하여 더 공부하고 싶다면 Special 1)을 공부하기를 권합니다.




\subsection{Function 1) 합성함수 이야기}
많은 학생들이 합성함수만 보면 경기를 일으키곤 합니다. 기본함수들의 특성과 이를 바탕으로 합성된 함수가 갖는 개략적인 특징을 알아보고, 삼각함수의 각변환을 전통적인 방법과 새로운 관점으로 해석하여 익혀봅니다. 그 후 $y=f\left( g\left( x \right)  \right) $의 그래프, $y=f\left( f\left( x \right)  \right)$의 그래프, 등식 $f\left( f\left( x \right)  \right) =x$의 해석을 차례로 배우며 합성함수에 대한 이해의 폭을 넓힙니다.

\subsection{Function 2) 역함수 이야기}
교과서의 실책으로 인해 많은 학생들이 역함수 개념에 혼동을 일으키고 있습니다. 깔끔하게 재정의된 역함수 관련 용어를 통해 역함수를 정확히 이해합시다. 이를 통해 $f\left( x \right) =f^{-1}\left( x \right) $의 해석, 역함수와 유사하지만 역함수가 아닌 유사 역함수 조건을 해석해봅시다. \cnm{미적분}학생들은 추가적으로 역함수의 미분법을 정확히 이해해봅시다.

\section{Calculus : 미적분 이론의 잠정적 완성}
Calculus는 네 중단원에 걸쳐 미적분을 본격적으로 배웁니다. 이를 통해 미적분 이론을 잠정적으로 완성하는 것을 목표로 합니다.

Calculus까지만 공부하더라도 사실상 수능을 대비하는 데 필요한 모든 개념을 배운 셈입니다. Special은 본인이 원한다면 추가학습하고, 아니라면 문제풀이를 통해 지금까지 배운 내용을 복습하며 개념편을 발췌독하기를 바랍니다.

\subsection{Calculus 1) 미적분의 융합과 그래프 그리기}
`정적분과 미분의 관계', `미적분의 기본 정리'를 이용하여 미분과 적분을 통합적으로 이해하고, 이를 바탕으로 Graph, Basic, Function에서 배운 내용을 다시 한 번 정당화합니다. 이를 다항함수 분석에 접목하여 삼차함수와 사차함수에 대한 완벽한 이해를 마쳐 장장 네 시즌에 걸친 다항함수 분석을 마무리합니다. 또한 미적분을 이용하여 그래프를 그리는 기본적인 원칙과, 그래프를 개략적으로 빠르게 그리는 편법을 배웁니다. 그 후 지금까지 배운 내용을 바탕으로 함수에 관한 여러가지 상황을 다루어보고, 마지막으로 미분계수의 정의와 도함수의 극한에 대하여 알아봅시다.

\subsection{Calculus 2) 원시함수를 구하는 테크닉}
적분만을 위한 단원입니다. Calculus 2.1)만 \cnm{미적분} 미선택 공통이고, 나머지는 모두 \cnm{미적분}입니다. \cnm{미적분} 학생들은 원시함수를 구하는 다양한 적분 테크닉을 기본적인 수준부터 응용까지 다룹니다. 단순히 `이럴 땐 이렇게 해야 한다'가 아닌, 적분 테크닉을 떠올리는 발상의 핵심 지점을 정확히 짚어주므로 적분을 바라보는 관점이 한층 업그레이드될 것입니다.

\subsection{Calculus 3) 음함수와 매개변수 길들이기}
음함수와 매개변수에 대해 무엇이 출제 가능하고 무엇이 불가능한지를 철저하게 따지고, 음함수와 매개변수, 특히 매개변수에 대해 탐구합니다. 

\subsection{Calculus 4) 물리학, 그리고 벡터}
위치, 속도, 가속도, 거리를 미적분으로 해석합니다. Calculus 4.1)에서는 교과 내의 설명으로 2~3페이지만에 깔끔하게 정리합니다. 4.1)만으로도 수능을 준비하는데는 아무런 문제가 없으므로 4.2)와 4.3)은 공부하지 않아도 무방합니다. 4.2)에서는 교과서의 허술한 서술을 보강하기 위해 벡터를 배우고, 4.3)에서는 벡터로 물리학을 해석함으로써 물리학에 대한 수학적 표현이 왜 그렇게 나타나는지를 명쾌하게 설명합니다.

\section{Special : 굳이 원한다면 주제별 특강을 통해 궁금증을 해결하자}
Special에서는 세 가지 특별한 주제를 다룹니다. 여기서 배우는 내용들은 많은 학생들이 공통적으로 궁금해하는 주제이다보니 매년 반복적으로 화제가 되는 이슈들입니다.

\subsection{Special 1) 역함수와 합성함수 확장팩}
Function의 심화학습을 위한 단원입니다. Special 1.1)에서는 역함수를 해석하기 위한 새로운 용어를 배우고, 이를 바탕으로 역함수 존재 조건과 유사 역함수 조건을 해석해봅니다. Special 1.2)는 \cnm{미적분} 전용으로, 매나함을 통해 합성함수를 해석하고, 벡터를 이용하여 합성함수의 그래프를 그리는 방법을 배웁니다. 

\subsection{Special 2) 볼록성과 접할선}
볼록성이 할선\mn{지금은 모르는 용어겠지만, 나중에 배우실 겁니다.}{}과 접선의 위치관계에 미치는 영향과 접선의 개수에 대한 내용을 단계적으로 배워봅니다.

\subsection{Special 3) 판도라의 상자}
도함수의 극한과 미분계수의 정의가 미묘하게 다르다는 점 때문에 빚어지는 오개념이 있습니다. 이 오개념을 모르고 편하고 행복한 삶을 살 것인지, 호기심을 이기지 못하고 진실이 궁금해 판도라의 상자를 열어젖힐 지는 학생 여러분의 선택입니다.

\end{comment}