\mychapter{여러 가지 식}{}
\section{등식}
등호와 수, 식을 이용하여 두 수, 수와 식, 식과 식이 서로 같음을 표현하는 식을 \term{등식}{}이라고 합니다.\mn{등식의 성질은 이 책에서 설명하지 않고 생략합니다.}{} 등식에는 항등식과 방정식이 있습니다. 방정식은 뒤에서 다루고, 먼저 항등식을 다루어 봅시다.
\subsection{항등식}
$x$에 대한 \term{항등식}{}은 식의 $x$에 대입할 수 있는 모든 실수 $x$에 대하여 항상 성립하는 등식입니다. 각 변이 $x$에 대한 다항식인 항등식을 정리하면 $0 = 0$꼴이 됩니다. 각 변이 $x$에 대한 다항식인 어떤 등식이 항등식이라면 각 변에서 차수가 같은 항의 계수가 서로 같음을 이용하거나, 식을 간단히 만드는 적당한 값을 대입하여 풀이하면 됩니다.


\section{다항식의 연산}
덧셈, 뺄셈, 곱셈(전개)의 방법에 대한 서술은 생략합니다. 이 방법을 모른다면 이 책을 공부하기를 권하지 않습니다.

\subsection{곱셈 공식과 인수분해 공식}
좌변에서 우변을 얻는 것이 \term{곱셈 공식}{}, 우변에서 좌변을 얻는 것이 \term{인수분해 공식}{}입니다.

\begin{enumerate}[label=\onum*]
    \item $(a+b)^2 = a^2 + 2ab + b^2$

    \item $(a-b)^2 = a^2 - 2ab + b^2$
    \item $(a+b)(a-b) = a^2 - b^2$
    \item $(x+a)(x+b)=x^2 + (a+b)x+ab$
    \item $(ax+b)(cx+d)=acx^2 + (ad+bc)x + bd$
    \item $(a+b)^3 = a^3 + 3a^2b + 3ab^2 + b^3$
    \item $(a-b)^3 = a^3 - 3a^2b + 3ab^2 - b^3$
    \item $(a+b)(a^2-ab+b^2) = a^3  + b^3$
    \item $(a-b)(a^2+ab+b^2)=a^3 - b^3$
    \item $(a+b+c)^2 = a^2 + b^2 + c^2 + 2ab + 2bc + 2ca$
\end{enumerate}\cleartorecto

\subsection{다항식의 나눗셈부터 인수정리까지}
앞서와 마찬가지로 나눗셈의 방법과 조립제법에 대한 서술은 생략합니다.
\subsubsection{나눗셈에 대한 항등식}
다항식 $A$를 $0$이 아닌 다항식 $B$로 나누었을 때의 몫을 $Q$, 나머지를 $R$라고 하면 다음이 성립합니다. \begin{align*} A=BQ+R \quad \text{ (단, $R$의 차수는 $B$의 차수보다 낮음)} \end{align*}
\subsubsection{나머지정리}
나눗셈에 대한 항등식을 이용하여 다항식 $f\left( x \right) $를 일차식 $x-a$로 나누었을 때의 몫을 $Q\left( x \right) $, 상수인 나머지를 $r$라고 하면 다음이 성립합니다. \begin{align*} f\left( x \right) =\left( x-a \right)Q\left( x \right)  +r\end{align*}
이때 양변에 $x=a$를 대입하면 $f\left( a \right) = r$입니다. 이를 \term{나머지정리}{}라고 합니다.

나머지정리는 `다항함수의 함숫값을 해석하는 새로운 관점'을 제시한다는 점에서 의미를 가집니다. 예를 들어, 기존에는 다항함수 $f\left( x \right) $의 함숫값 $f\left( 3 \right) $을 구하기 위해서는 다항식에 포함된 각 항에 $x=3$을 대입하여 그 값을 계산해야 했고, 그 값에서 특별한 의미를 찾기 힘들었습니다. 그러나 이제는 나머지를 이용해 다른 함수를 $y$축 방향으로 평행이동한 것으로 해석할 수 있습니다. 이는 함수의 극한이나 미적분과 연계되어 다양한 해석과 풀이의 실마리를 제공하기도 합니다.

\subsubsection{인수정리}
나머지정리에서 $r=0$이면, 즉 $f\left( a \right) =0$이면 $f(x)$가 $x-a$로 나누어떨어집니다. 반대로, $f(x)$가 $x-a$로 나누어떨어지면 $r=0$이므로 $f\left( a \right) =0$입니다. 이를 \term{인수정리}{}라 합니다.

인수정리는 `다항식을 세련된 방식으로 인수분해하는 방법'을 제시한다는 점에서 의미를 가집니다. 이 또한 함수의 극한이나 미적분과 연계되어 다양한 해석과 풀이의 실마리를 제공하기도 합니다.
\clearpage
\section{방정식}
$x$의 값에 따라 참이 되기도 하고 거짓이 되기도 하는 $x$에 대한 등식을 방정식이라고 합니다. 이때 $x$를 \term{미지수}{}라 하며, 이 방정식을 참이 되게 하는 $x$를 \term{해}{} 또는 \term{근}{}이라고 합니다.

\subsection{일차방정식}
미지수에 대한 일차 다항식으로 구성된 방정식을 \term{일차방정식}{}이라고 합니다. 적절히 이항하여 근을 구할 수 있습니다.

\subsection{이차방정식의 근과 판별식}
미지수에 대한 이차 다항식으로 구성된 방정식을 \term{이차방정식}{}이라고 합니다. 각 항의 계수가 실수인 이차방정식 $ax^2 + bx + c=0\: (a\ne 0)$의  두 근은 다음과 같습니다.
\begin{align*} x= \dfrac{-b\pm\sqrt{b^2-4ac}}{2a} \end{align*}
이때 $b^2 - 4ac \ge 0$이면 $\sqrt{b^2 - 4ac}$는 실수이고, $b^2 - 4ac < 0$이면 허수\mn{허수 및 복소수는 교육과정 미적분에서 비중이 매우 떨어지므로 Zero에 수록하지 않았습니다.}{}입니다. 따라서 계수가 실수인 이차방정식은 복소수의 범위에서 반드시 근을 가지며, 실수인 근을 \term{실근}{}, 허수인 근을 \term{허근}{}이라고 합니다. 이와 같이 $b^2 - 4ac$는 이차방정식의 근을 판별해주는 역할을 하므로 \term{판별식($D$)}{}이라 합니다.

\subsection{고차방정식과 연립방정식}
미지수에 대한 삼차 이상의 다항식으로 구성된 방정식을 \term{고차방정식}{}이라 합니다. 인수분해가 가능한 식은 인수분해하여 풀이할 수 있습니다. 그렇지 않은 고차방정식은 미분을 배운 후 제한적으로 풀이할 수 있습니다.

둘 이상의 방정식을 연립한 것을 \term{연립방정식}{}이라고 합니다. 연립방정식은 대입/가감 등을 이용하여 미지수를 적절히 소거하여 풀이합니다.
\clearpage
\section{부등식}
부등호 $<$, $>$, $\le$, $\ge$을 이용하여 두 수, 수와 식, 식과 식의 크기를 비교하는 식을 \term{부등식}{}이라고 합니다. 어떤 부등식이 마치 방정식처럼 $x$의 값에 따라 참이 되기도 하고 거짓이 되기도 할 때, 그 부등식을 `\term{$x$에 대한 부등식}{}'이라고 합니다. 이때 $x$를 \term{미지수}{}라 하며, 이 부등식을 참이 되게 하는 $x$를 \term{해}{}라고 합니다.\mn{방정식에서는 근과 해가 두루 쓰이지만, 부등식에서는 근이라는 표현을 거의 쓰지 않습니다.}{}

\subsection{일차부등식}
적절히 이항하여 해를 구할 수 있습니다. 

\subsection{이차부등식의 근과 판별식}
이차부등식의 풀이는 단순 암기에 그치면 안 되고, 수식과 그래프를 넘나드는 복합적 이해를 통해 숙지해야 합니다. 따라서 Graph 0)에서 다룹니다.

\subsection{고차부등식과 연립부등식}
삼차 이상의 부등식을 \term{고차부등식}{}이라 합니다. 인수분해가 가능한 식은 인수분해하여 풀이할 수 있습니다. 그렇지 않은 고차부등식은 미분을 배운 후 제한적으로 풀이할 수 있습니다.

둘 이상의 부등식을 연립한 것을 \term{연립부등식}{}이라고 합니다. 연립부등식은 각각의 부등식의 해를 구한 후, 모든 부등식의 공통해를 취하여 풀이합니다.


\mychapter{명제}{}

\section{명제}
참인지 거짓인지를 분명하게 판별할 수 있는 문장이나 식을 \term{명제}{}라고 합니다. 예를 들어 `$6$은 $2$의 배수이다'라는 문장과 $2 - 4 = 0$이라는 식은 명제입니다.

\section{조건}
`$x$는 $2$의 배수이다'라는 문장과 $x-4 = 0$이라는 식은 그 자체로는 명제가 아니지만, $x$의 값에 $4$를 대입하면 각각 참, 참이고, $x$의 값에 $6$을 대입하면 각각 참, 거짓이고, $x$의 값에 $5$를 대입하면 각각 거짓, 거짓입니다. 이와 같이 변수를 포함하는 문장이나 식이 변수의 값에 따라 참인지 거짓인지 결정될 때, 이러한 문장이나 식을 \term{조건}{}이라고 합니다.

\subsection{조건의 진리집합}
어떤 전체집합 $U$가 주어졌을 때, $U$의 원소 중에서 $p$라는 조건을 참이 되도록 하는 원소의 집합을 조건 $p$의 진리집합이라고 합니다. 집합의 조건제시법에서, 조건 $p$를 제시했을 때 얻는 집합이 바로 진리집합입니다.

\section{명제의 부정}
명제 $p$에 대하여 `$p$가 아니다'라는 명제를 `명제 $p$의 부정'이라고 하며 $\sim p$라 표기합니다. $p$가 참이면 $\sim p$는 거짓이고, $p$가 거짓이면 $\sim p$는 참입니다.

조건 $q$에 대하여 `$q$가 아니다'라는 조건을 `조건 $q$의 부정'이라고 하며 $\sim q$라 표기합니다. 전체집합 $U$에 대하여 $q$의 진리집합이 $Q$이면 $\sim q$의 진리집합은 $Q^C$입니다.

\section{`모든'과 `어떤'}

\subsection{`모든'을 포함한 명제}
`모든 $x$에 대하여 $p$이다'가 참이라는 것은 전체집합 $U$의 모든 원소 $x$에 대하여 조건 $p$가 참이라는 것(조건을 만족시킨다는 것)을 뜻합니다. `모든 $x$에 대하여 $p$이다'가 거짓이라는 것은 전체집합 $U$의 원소 중 조건 $p$를 참이 되게 하지 않는 원소가 적어도 하나 존재한다는 것을 뜻합니다. 따라서 조건 $p$의 진리집합이 $P$일 때, $P=U$이면 주어진 명제가 참이고, $P \ne U$이면 주어진 명제가 거짓입니다.

\subsection{`어떤'을 포함한 명제}
`어떤 $x$에 대하여 $p$이다'가 참이라는 것은 전체집합 $U$의 원소 중 조건 $p$를 참이 되게 하는 원소가 적어도 하나 존재한다는 것을 뜻합니다. `어떤 $x$에 대하여 $p$이다'가 거짓이라는 것은 전체집합 $U$의 원소 중 조건 $p$가 참이 되게 하는 원소가 하나도 존재하지 않는다는 것을 뜻합니다. 따라서 조건 $p$의 진리집합이 $P$일 때, $P \ne \emptyset$이면 주어진 명제가 참이고, $P =\emptyset$이면 주어진 명제가 거짓입니다.
\cleartorecto
\subsection{`모든'과 `어떤'의 관계}
`모든 $x$에 대하여 $p$이다'의 부정은 `어떤 $x$에 대하여 $\sim p$이다'이고, `어떤 $x$에 대하여 $p$이다'의 부정은 `모든 $x$에 대하여 $\sim p$이다'입니다.


\section{가정과 결론으로 구성된 명제}
진리집합이 각각 $P$, $Q$인 두 조건 $p$와 $q$를 `$p$이면 $q$이다.'의 꼴로 연결한 명제를 $p \longrightarrow q$라 표기하고, $p$를 가정, $q$를 결론이라고 합니다. $p \longrightarrow q$가 참이면 $P \subset Q$입니다. 반대로 $P \subset Q$이면 $p \longrightarrow q$는 참입니다.  $p \longrightarrow q$가 거짓이면 $P \not\subset Q$입니다. 반대로 $P \not\subset Q$이면 $p \longrightarrow q$는 거짓입니다.

\subsection{명제의 역과 대우}
명제 $q \longrightarrow p$를 $p \longrightarrow q$의 역이라 합니다. 명제 $\sim q \longrightarrow \sim p$를 $p \longrightarrow q$의 대우라고 합니다. 명제가 참이면 대우도 참입니다.\mn{진리집합의 포함관계를 통해 알 수 있습니다.}{} 명제가 참이라고 해서 역이 참인 것은 아닙니다(참인 경우도 있고, 거짓인 경우도 있습니다).

\section{정의, 증명, 정리}
용어의 뜻을 명확하게 정한 문장을 정의라 합니다. 정의 혹은 이미 알려진 사실이나 성질을 이용하여 명제가 참 또는 거짓임을 밝히는 과정을 증명이라고 합니다. 또한 참으로 증명된 명제 중에서 기본이 되는 것, 여러 가지 성질을 증명할 때 자주 이용되는 것을 정리라고 합니다. 어떤 명제를 증명할 때에는 명제를 가정과 결론으로 나누어 생각하면 편리합니다.

\subsection{대우를 이용한 증명법}
명제를 증명하기 어려운 경우, 대우를 이용하여 증명하면 편리한 경우가 있습니다.

\subsection{귀류법}
가정에서 결론을 직접 이끌어내기 어렵지만, 명제를 부정하면 모순이 발생함을 보이는 것은 쉬운 경우가 있습니다. 이와 같이 명제의 부정에서 모순을 이끌어내어 원래 명제가 참임을 보이는 증명 방법을 \term{귀류법}{}이라고 합니다.

\section{필요조건과 충분조건}
$p \longrightarrow q$가 참이면 $p \Longrightarrow q$라 표기합니다. 이때 $p$는 $q$이기 위한 충분조건, $q$는 $p$이기 위한 필요조건이라고 합니다. 

$p \Longrightarrow q$이고 $q \Longrightarrow p$일 때, $p \Longleftrightarrow q$라 표기하고, $p$는 $q$이기 위한 필요충분조건이라고 합니다. 이때 두 조건의 진리집합은 서로 같습니다. 



